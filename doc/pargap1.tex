%%%%%%%%%%%%%%%%%%%%%%%%%%%%%%%%%%%%%%%%%%%%%%%%%%%%%%%%%%%%%%%%%%%%%%%%%
%%
%W  pargap1.tex            ParGAP documentation            Gene Cooperman
%%
%Y  Copyright (C) 1999-2001  Gene Cooperman
%Y    See included file, COPYING, for conditions for copying
%%

\pretolerance=500 % Will tolerate badness of 500 before trying hyphenations%
\tolerance=1600 % Will tolerate stretching line up to badness of 1600%
\hbadness=4000 % Seems to affect overfull boxes reported by TeX%
\hfuzz=5pt % If still no good break, can stick out into margin by 5 pt.%
\overfullrule=0pt % Lines sticking out more than 10 pt should not%
                  % contain the black box marking it.%

\Chapter{Writing Parallel Programs in GAP Easily}

\indextt{ParGAP}
The {\ParGAP}  (Parallel  {\GAP})  package  provides  a  way  of  writing
parallel programs using the {\GAP} language. Former names of the  package
were \package{ParGAP/MPI} and \package{GAP/MPI}; the word <MPI> refers to
<Message Passing  Interface>,  a  well-known  standard  for  parallelism.
{\ParGAP} is based on the MPI standard, and this distribution includes  a
subset implementation of MPI, to provide a portable  layer  with  a  high
level interface to BSD sockets. Since knowledge of MPI  is  not  required
for use of  this  software,  we  now  refer  to  the  package  as  simply
{\ParGAP}. For more information visit the author's  {\ParGAP}  home  page
at:
\URL{http://www.ccs.neu.edu/home/gene/pargap.html}

For some background reading, see~\cite{Coo95} and \cite{Coo97}.

This first chapter is intended to help a new user set  up  {\ParGAP}  and
run through some quick examples: see

\beginlist%unordered

\item{$\bullet$}
Section~"Overview of ParGAP" for an overview of the features of {\ParGAP}
and a general discussion of how it's implemented;

\item{$\bullet$}
Section~"Installing ParGAP" for how to install {\ParGAP};

\item{$\bullet$}
Section~"Running ParGAP"  for  how  to  run  {\ParGAP}  (*not*  by  using
`LoadPackage'); and

\item{$\bullet$}
Section~"Extended Example"  for  some introductory {\ParGAP} examples.

\endlist

The later chapters present detailed explanations  of  the  facilities  of
{\ParGAP}. Because parallel programming is  sufficiently  different  from
sequential programming, this author  recommends  printing  out  at  least
Chapters~1 through~"MasterSlave Tutorial",  and  skimming  through  those
chapters for areas of interest, before returning to the terminal  to  try
out   some   of   the   ideas.   This   document   can   be   found    in
`.../pkg/pargap/doc/manual.dvi' of the  software  distribution.  You  may
also want to print the index at the end of `manual.dvi'.  In  particular,
the heading `example' in the index, or `??example'  from  within  {\GAP},
should be useful. If you prefer postscript, the UNIX command `dvips' will
convert that file to postscript form.

The development of {\ParGAP} was partially supported by National  Science
Foundation grants CCR-9509783 and CCR-9732330.

%%%%%%%%%%%%%%%%%%%%%%%%%%%%%%%%%%%%%%%%%%%%%%%%%%%%%%%%%%%%%%%%%%%%%%%%%
\Section{Overview of ParGAP}

{\ParGAP} is installed on top of an existing {\GAP} installation. It 
comes with its own subset MPI implementation (currently functional only on 
UNIX installations), or it can use your system MPI libraries, if present. 
See Section~"Installing  ParGAP" for instructions on installation of {\ParGAP}. 
At the time that {\ParGAP} is invoked, a special file or command line 
parameter must be used to tell {\ParGAP}  how many local processes or which 
remote machines to use for slave processors. See section~"Running ParGAP" for 
instructions on invoking {\ParGAP}. If there are  questions  or  bugs
concerning {\ParGAP}, please write to: \Mailto{gene@ccs.neu.edu}

If one wishes only to try out the parallel features, the first five pages
of this manual (through the section on the slave listener)  will  suffice
for installation, and using it. For the more advanced user who wishes  to
design new parallel algorithms or port old sequential code to a  parallel
environment, it  is  strongly  recommended  to  also  read  the  sections
following  on  from  Section~"Basic  Concepts   for   the   TOP-C   model
(MasterSlave)".

{\ParGAP} should be invoked via the script `bin/pargap.sh' created by the
installation process which invokes `<GAP_ROOT_DIR>/bin/<ARCH>/pargapmpi',
where <ARCH> depends on your system but is the same  directory  in  which
the `gap' binary is  found.  MPI  and  the  higher  layers  will  not  be
available if the binary is invoked in the standard way as `gap'. This  is
a feature, since a single binary and source distribution serves both  for
the standard {\GAP} and for {\ParGAP}.

{\ParGAP} is implemented in three layers: 1)~MPI, 2)~Slave~Listener,  and
3)~Master~Slave (TOP-C abstraction). Most users will find  that  the  two
highest layers (Slave Listener and Master Slave) meet all their needs.

\beginitems
`1) MPI:'&
    The lowest layer is MPI. Most users can ignore this layer. MPI  is  a
    standard for message-based parallel  computation.  A  subset  of  the
    original MPI commands is exposed at the {\GAP} level. The syntax is
    modified from the original C binding to  make  a  {\GAP}  binding in
    an interpreted environment more convenient. This includes default
    arguments, useful return values, and `Error' break in the presence of
    errors. `MPI_Init()' (see~"MPI_Init") and `MPI_Finalize()'
    (see~"MPI_Finalize") are invoked automatically by {\ParGAP}.

`'& The MPI layer is not documented, since most users will not  be  using
    it. From {\GAP} level, you can  type:  `MPI_<tab><tab>'  to  see  all
    implemented MPI functions and variables. However, typing  the  symbol
    name alone (e.g.: `MPI_Send;' ) will cause it to display the  calling
    syntax. The same information is displayed after  an  incorrect  call.
    The  return  value  is  typically  obvious.  MPI  is  implemented  in
    `src/pargap.c'. {\ParGAP} will use a sysem MPI implementation if one is
    present, and the distribution also includes two versions of a simple, subset
    implementation of MPI in `pkg/gapmpi/mpinu/' and `pkg/gapmpi/mpinu2/', 
    which is implemented on top of a standard sockets interface, which can be 
    used instead..

\atindex{MPI!standard}{@MPI!standard}
`'& For those who wish to directly use the MPI interface, the meanings of
    the MPI calls are best found from the standard MPI documentation:

`'&MPI Forum: \URL{http://www.mpi-forum.org/}

`'&MPI Standard (version 1.1):
   \URL{http://www.mpi-forum.org/docs/mpi-11-html/mpi-report.html}

`'&UNIX style man pages: \URL{http://www.mcs.anl.gov/research/projects/mpi/www/}

`2) Slave Listener:'&
    This  layer   provides   basic   message   passing   facilities   for
    communication among multiple {\ParGAP} processes in a  form  that  is
    more convenient for programming than the lower MPI layer.  This  will
    be the most useful entry point to {\ParGAP} for most users.  This  is
    the default mode for {\ParGAP}. Each remote (slave) process is  in  a
    receive-eval-send loop, in which the slave receives a {\GAP}  command
    from the local or master, the slave evaluates the {\GAP} command, and
    the slave then sends the result  back  to  the  master  as  a  {\GAP}
    object.

`'&
    Almost all commands in the slave listener are  of  the  form  `*Msg*'
    e.g.  `SendMsg()'   (see~"SendMsg"),   `RecvMsg()'   (see~"RecvMsg"),
    `ProbeMsg()'   (see~"ProbeMsg").   Since   the   slave   is   in    a
    receive-eval-send loop, every `SendMsg(<cmd>)' on the master must  be
    balanced by a later `RecvMsg()'. `SendRecvMsg()'  (see~"SendRecvMsg")
    is provided to combine these steps. A few parallel utilities are also
    included, such as `ParRead()' ("ParRead"),  `ParList()'  ("ParList"),
    `ParEval()' ("ParEval"), etc.

`'& Messages are arbitrary {\GAP} objects. Note  that  arguments  to  any
    {\GAP} function are evaluated before being passed  to  the  function.
    Hence, any argument to `SendMsg()' or `ParEval()' would be  evaluated
    locally before being  sent  across  the  network.  For  this  reason,
    arguments can also be given as strings,  to  delay  evaluation  until
    reaching the destination process. Hence, real strings must be quoted:
    `ParEval("x:=\"abc\";");' Additionally, multiple commands are  valid,
    and the final ```;''' of the string is optional. So, one can write:

\begintt
BroadcastMsg("x:=\"abc\"; Print(Length(x), \"\\n\")");;
\endtt

`'& A full description is contained in Chapter~"Slave Listener".

`3) Master Slave:'&
    The Master Slave  facility  is  provided  both  for  writing  complex
    parallel software, and as an easier way to  parallelize  previous  or
    ``legacy''  sequential  code.  While  the  Slave  Listener   may   be
    sufficient for simple parallel requirements,  more  complex  software
    requires a higher level abstraction. The fundamental abstractions  of
    the master slave layer are the *task* and the *shared data*.

\beginlist
\itemitem{`1)'}
    The task typically corresponds to the procedure or inner  body  of  a
    loop in  a  sequential  program.  This  is  the  part  that  must  be
    repetitively computed in parallel.

\itemitem{`2)'}
    The shared data typically corresponds to the  data  of  a  sequential
    program that is not within the local scope of the task. Often this is
    a global data structure. In the case that the task is the inner  body
    of a loop, the shared data may be a  local  data  structure  that  is
    outside the local scope of the loop.
\endlist

`'& It is usually quite easy to identify the task and the shared data  of
    a sequential program  or  algorithm,  which  is  the  first  step  in
    parallelizing an algorithm.

`'& The  Master  Slave  parallel  model  described  here  has  also  been
    successfully used in~C  and  in  LISP.  It  has  been  used  both  in
    distributed memory and  shared  memory  environments,  although  this
    version in {\GAP} currently works only in a distributed  environment.
    In the C~language, this  parallel  model  is  known  as  TOP-C  (Task
    Oriented Parallel~C). For examples of the use of the TOP-C model  see
    \cite{Coo98},     \cite{CCHW02},      \cite{CFTY94},     \cite{CG02},
    \cite{CH97}, \cite{CHLM97}, \cite{CLMW96}, and \cite{CT96}.

`'& While no parallel software can eliminate the problem of designing  an
    algorithm that is efficient in  a  parallel  environment,  the  TOP-C
    abstraction eases the job by eliminating  programmer  concerns  about
    lower  level  details,  such  as  message  passing,   migration   and
    replication of data, load balancing, etc. This leaves the  programmer
    to concentrate on the primary goal:  maximizing  the  concurrency  or
    parallelism.

\enditems

%%%%%%%%%%%%%%%%%%%%%%%%%%%%%%%%%%%%%%%%%%%%%%%%%%%%%%%%%%%%%%%%%%%%%%%%%
\Section{Choosing an MPI Library}

If you are using Linux and wish to try out {\ParGAP} quickly, you
can skip this section and let the {\ParGAP} build process choose an MPI
library for you. If you have a little more time, or are running on a different
system, please read on.

{\ParGAP} uses MPI, a standard Message Passing  Interface for 
communicating between processes. Since the details of inter-process 
communication are system-specific, {\ParGAP} relies on an external library
to provide its MPI functions. A implementation of a sufficient subset of MPI,
which runs on Linux and OS X, is included with {\ParGAP}. Alternatively, an MPI
library can be installed on your system before building {\ParGAP}. Two popular
MPI implementations are:
\beginlist
\item{} MPICH2 \URL{http://www.mcs.anl.gov/research/projects/mpich2/}
\item{} Open MPI \URL{http://www.open-mpi.org/}
\endlist
Both of these are compatible with Linux, Macs and Windows. Installation packages
can be downloaded from their websites, or may be available through your systems
standard package management mechanism.

The MPINU library included with {\ParGAP} provides the MPI functionality that
{\ParGAP} needs by using Unix sockets. This implementation is sufficient for
basic {\ParGAP} usage, but does not scale to larger systems as well as the 
alternative system libraries. It is better-suited to interative {\ParGAP} 
sessions, since system MPI implementations can result in problems with line 
editing in {\ParGAP}. When built with MPINU, {\ParGAP} also enables two 
commands `ParReset()' and `FlushAllMsgs()' which can be useful when developing 
parallel programs. See 
Section~"Problems Running ParGAP with a System MPI Implementation" for details
of these known issues with system MPI implementations. Two versions of MPINU
are included with {\ParGAP}: the original MPINU and a newer version, called 
MPINU2. 

On Linux machines, we recommend that you use {\ParGAP} with a system MPI 
implementation instead of MPINU, if possible. These implementations provide 
better performance and fault tolerance, and are compatible with a wider range 
of operating systems and hardware, including high speed networks and 
proprietory high-end computing systems. 

On Macs, we recommend using the original MPINU since there are currently some 
problems running {\ParGAP} with both a system MPI implementation and MPINU2.
Both these issues will hopefully be resolved in a future release.

By default, the {\ParGAP} build process 
(see Section~"Installing ParGAP") tries to use a system MPI implementation if 
it can find one. If not, it will use MPINU. Two versions of MPINU are included 
with this release of {\ParGAP}. The recommended choice is MPINU2, but the 
original MPINU is included as a backup in case there are problems building or 
running MPINU2.


%%%%%%%%%%%%%%%%%%%%%%%%%%%%%%%%%%%%%%%%%%%%%%%%%%%%%%%%%%%%%%%%%%%%%%%%%
\Section{Installing ParGAP}

\index{installation}
Installing {\ParGAP} should be relatively simple.  However,  since  there
are many interactions both with the  {\GAP}  kernel  and  with  the  UNIX
operating system, in a minority of cases,  manual  intervention  will  be
necessary. If you are part of  this minority,  please  see  the  
section~"Problems Installing or Invoking ParGAP". The most common problem 
is the local security policy; {\ParGAP} is more pleasant to use when  you  
don't have  to  manually  provide the password for  each  slave.  See  
section~"Problems  with Passwords (Getting Around Security)" for  
suggestions  in this respect.

To install the {\ParGAP} package, move  the  file  `pargap-<XXX>.zoo'  or
`pargap-<XXX>.tar.gz' (for some version number <XXX> of  {\ParGAP})  into
the `pkg' directory in which you plan to install {\ParGAP}. Usually, this
will be the directory `pkg' in the hierarchy of your version of  {\GAP}
(in fact, currently it is  not  possible  to  have  the  `pkg'  directory
separate from {\GAP}'s `pkg' directory; we hope to remedy this in  future
versions of {\ParGAP} so that it will also possible to keep an additional
`pkg' directory in your private directories; section "ref:Installing a GAP Package" 
of the GAP reference manual gives details on how to do  this,
when it's possible.)

Now change into  the  `pkg'  directory  in  which  you  plan  to  install
{\ParGAP}. If you got a `.zoo' file, unpack it with:

\){\kernttindent}unzoo -x pargap-<XXX>

If you got a `.tar.gz' file and  your  `tar'  command  supports  the  `z'
option, unpack it with:

\){\kernttindent}tar zxf pargap-<XXX>.tar.gz

or otherwise unpack in two steps with:

\){\kernttindent}gunzip pargap-<XXX>.tar
\){\kernttindent}tar xvf pargap-<XXX>.tar

Whether you got the `.zoo' or `.tar.gz' archive you should now have a new
directory `pargap'. As for a generic {\GAP} package, do:

\begintt
cd pargap
./configure
make
\endtt

This builds the {\ParGAP} files. {\ParGAP} also needs to rebuild parts of 
{\GAP} to enable the MPI hooks. It may also need to re-run the {\GAP} 
`configure' if you have a dedicated MPI compiler. By default, the {\ParGAP}
`configure' will prompt you to do this by hand if necessary, and then to 
restart the {\ParGAP} build. If you are happy for the {\ParGAP} build process
to run the {\GAP} `configure' for you if needed, with no arguments, then run 
{\ParGAP}'s `configure' with
\begintt
./configure --with-basic-gap-configure
\endtt

The `configure' script will attempt to find a system MPI implementation that
it can use. If if not then it will use MPINU2, the more recent of the two 
MPINU subset implementations included with the {\ParGAP} package. You can use 
the `--with-mpi=' configure option to specify a different behaviour, and you 
can also set your own MPI compiler and options if you wish. See the help text 
provided by  `./configure -h' for full details.

After doing the `configure' and `make' steps of {\ParGAP}'s  installation
process (see Section~"Installing ParGAP"), you should find in {\ParGAP}'s
`bin' subdirectory a script

\begintt
pargap.sh
\endtt

which you should use to start {\ParGAP}. ({\ParGAP} can *not* be  started
by starting {\GAP}~4 in the usual way, and using `LoadPackage';  doing
so will result in `Info'-ed  advice  to  read  this  section.)  Edit  the
`pargap.sh' script if necessary, copy it to a standard path and rename it
according to how you intend to call {\ParGAP} (e.g. rename it: `pargap').

*Note:*
The script  `pargap.sh'  defines  the  program  that  runs  {\ParGAP}  as
`pargapmpi'. In fact, after installation `pargapmpi' is a  symbolic  link
to the {\GAP} binary named `gap'. The same binary runs  both  {\GAP}  and
{\ParGAP}; when the binary is invoked as `gap' {\GAP} runs in  the  usual
way without any parallel features; only when the  binary  is  invoked  as
`pargapmpi'    are    the    parallel    features    incorporated.    See
Section~"Modifying the GAP kernel" for more details.

Your {\ParGAP} should now be ready to use.  Now  read  the  next  section
which decribes how to  run  {\ParGAP}  (if  you  are  reading  this  from
{\GAP}'s on-line help, type: `?>').

%%%%%%%%%%%%%%%%%%%%%%%%%%%%%%%%%%%%%%%%%%%%%%%%%%%%%%%%%%%%%%%%%%%%%%%%%
\Section{Running ParGAP}

After a successful build, you will see a message saying that {\ParGAP} is
ready to use, and confirmation of whether a system MPI library or MPINU will 
be used. The method of running {\ParGAP} depends on this MPI choice, and the 
MPI library is auto-detected, or can be specified, in `configure', as 
described in Section~"Installing ParGAP". The pros and cons of the two 
different library variants are discussed in Section~"Choosing an MPI Library".

We will assume that you have copied the `pargap.sh' script to a location
on your search path and renamed it as `pargap', as suggested in 
Section~"Installing ParGAP".

*If you are using a system MPI library:*
{\ParGAP} should be started using an MPI launcher script. The name and syntax
of the command to start MPI processes can vary, and you should check your 
system MPI documentation for details. However, one common launcher is 
`mpiexec', and the following command should work with both Open MPI and MPICH,
and most other MPI-2 implementations:

\begintt
mpiexec -n 3 pargap
\endtt

This will start three copies of the {\ParGAP}: one master and two slaves. These
processes will all run on your local machine. See 
Section~"Invoking ParGAP with Remote Slaves (when using a system MPI library)"
for how to configure and run processes on remote slaves.

*If you are using MPINU:*
In {\ParGAP}'s `bin' subdirectory you should find a `procgroup' file which
defines the master and slave processes that will be used  by  {\ParGAP}.
When {\ParGAP} is started, the MPINU library looks for a file called `procgroup'  
in the current directory, unless the `-p4pg' option is used. Thus if you renamed
your shell script `pargap', the following  are  valid  ways  of  starting
{\ParGAP}:

\begintt
pargap
\endtt

(if current directory contains the file: `procgroup'), or

\){\kernttindent}pargap -p4pg <myprocgroupfile>

(where <myprocgroupfile> is the complete path of your  procgroup  file --
there is no restriction on how you name it). The default `procgroup' file
defines one master and two slaves on the local machine. For instructions of 
how to run remote slaves, see 
Section~"Invoking ParGAP with Remote Slaves (when using MPINU)".

If you had trouble installing or starting {\ParGAP}, see the 
section~"Problems Installing or Invoking ParGAP". Otherwise you are ready 
to test your installation, Try the example in the following section (if you 
are reading this from  {\GAP}'s on-line help, type: `?>').

%%%%%%%%%%%%%%%%%%%%%%%%%%%%%%%%%%%%%%%%%%%%%%%%%%%%%%%%%%%%%%%%%%%%%%%%%
\Section{Extended Example}

After  installation,  try  it  out.  Invoke  {\ParGAP}  as  described  in
Section~"Running ParGAP" and try the example below (but  substitute  your
own program where you see `"/home/gene/myprogram.g"').  The  commands  in
this first example are also found in the `README' file. So, you may  wish
to copy text from the `README' file and paste it into a `ParGAP' session.
If you have not specified any additional machines to the MPI launcher, or you
are using the unmodified `procgroup' file, then your *remote slaves*
will be other processes on your local machine. It is a good idea  to  run
only on your local machine for your first experiments and while  you  are
debugging parallel programs. When  you  wish  to  experiment  with  using
remote machines, you can then proceed to 
section~"Invoking ParGAP with Remote Slaves (when using a system MPI library)" 
or section~"Invoking ParGAP with Remote Slaves (when using MPINU)" depending
on which MPI library {\ParGAP} has been built to use.

\atindex{example!Slave Listener}{@example!Slave Listener}
\atindex{Slave Listener!example}{@Slave Listener!example}
\beginexample
gap> # This assumes your procgroup file includes two slave processes.
gap> PingSlave(1); #a `true' response indicates Slave 1 is alive
true
gap> # Print() on slave appears on standard output 
gap> # i.e. after the master's prompt.
gap> SendMsg( "Print(3+4)" );
gap> 7
gap> # A <return> was input above to get a fresh prompt.
gap> #
gap> # To get special characters (including newline: `\n')
gap> # into a string, escape them with a `\'.
gap> SendMsg( "Print(3+4,\"\\n\")" );
gap> 7

gap> # Again, a <return> was input above after the 7 and new-line
gap> # were printed to get a fresh prompt.
gap> #
gap> # Each SendMsg() is normally balanced by a RecvMsg().
gap> SendMsg( "3+4", 2);
gap> RecvMsg( 2 );
7
gap> # The following is equivalent to the two previous commands.
gap> SendRecvMsg( "3+4", 2);
7
gap> # The two SendMsg() commands that were sent to Slave 1 earlier have
gap> # responses that are waiting in the message queue from that slave.
gap> # Check that there is a message waiting. With some MPI implementations
gap> # the message is not immediately available, but when ProbeMsg() does
gap> # return true then RecvMsg() is guaranteed to succeed. 
gap> ProbeMsgNonBlocking( 1 );
false
gap> ProbeMsgNonBlocking( 1 );
true
gap> # Print() is a `no-value' functions, and so the result of a RecvMsg() 
gap> # in both these cases is "<no_return_val>".
gap> RecvMsg( 1 );
"<no_return_val>"
gap> RecvMsg( 1 );
"<no_return_val>"
gap> # As with Print() the result of Exec() appears on standard
gap> # output, and the result is "<no_return_val>".
gap> SendRecvMsg( "Exec(\"pwd\")" ); # Your pwd will differ :-)
/home/gene
"<no_return_val>"
gap> # Define a variable on a slave
gap> SendRecvMsg( "a:=45; 3+4", 1 );
7
gap> # Note "a" is defined on slave 1, not slave 2.
gap> SendMsg( "a", 2 ); # Slave prints error, output on master
gap>  Variable: 'a' must have a value
gap> # <return> entered to get fresh prompt.
gap> RecvMsg( 2 ); # No value for last SendMsg() command
"<no_return_val>"
gap> RecvMsg( 1 );
45
gap> # Execute analogue of GAP's List() in parallel on slaves.
gap> squares := ParList( [1..100], x->x^2 );
[ 1, 4, 9, 16, 25, 36, 49, 64, 81, 100, 121, 144, 169, 196, 225, 256, 
  289, 324, 361, 400, 441, 484, 529, 576, 625, 676, 729, 784, 841, 
  900, 961, 1024, 1089, 1156, 1225, 1296, 1369, 1444, 1521, 1600, 
  1681, 1764, 1849, 1936, 2025, 2116, 2209, 2304, 2401, 2500, 2601, 
  2704, 2809, 2916, 3025, 3136, 3249, 3364, 3481, 3600, 3721, 3844, 
  3969, 4096, 4225, 4356, 4489, 4624, 4761, 4900, 5041, 5184, 5329, 
  5476, 5625, 5776, 5929, 6084, 6241, 6400, 6561, 6724, 6889, 7056, 
  7225, 7396, 7569, 7744, 7921, 8100, 8281, 8464, 8649, 8836, 9025, 
  9216, 9409, 9604, 9801, 10000 ]
gap> # Send a large, local (non-remote) data structure to a slave
gap> Concatenation("x := ", PrintToString([1..10]*2));
"x := [ 2, 4, 6, 8, 10, 12, 14, 16, 18, 20 ]\n\000"
gap> SendMsg( Concatenation("x := ", PrintToString([1..10]*2)) ); 
gap> RecvMsg();
[ 2, 4, 6, 8, 10, 12, 14, 16, 18, 20 ]
gap> # Send a local (non-remote) function to a slave
gap> myfnc := function() return 42; end;;
gap> # Use PrintToString() to define myfnc on all slave processes
gap> BroadcastMsg( PrintToString( "myfnc := ", myfnc ) );
gap> SendRecvMsg( "myfnc()", 1 );
42
gap> # Ensure problem shared data is read into master and slaves.
gap> # Try one of your GAP program files instead.
gap> ParRead( "/home/gene/myprogram.g");
\endexample

Now that you have done a fairly rudimentary test of {\ParGAP} you  should
be ready to do something a little bit more interesting:

\beginexample
gap> ParInstallTOPCGlobalFunction( "MyParList",
> function( list, fnc )
>   local result, iter;
>   result := [];
>   iter := Iterator(list);
>   MasterSlave( function() if IsDoneIterator(iter) then return NOTASK;
>                           else return NextIterator(iter); fi; end,
>                fnc,
>                function(input,output) result[input] := output;
>                                       return NO_ACTION; end,
>                Error
>              );
>   return result;
> end );
gap> MyParList( [1..25], x->x^3 );
master -> 1:  1
master -> 2:  2
2 -> master: 8
1 -> master: 1
master -> 1:  3
master -> 2:  4
2 -> master: 64
1 -> master: 27
master -> 1:  5
master -> 2:  6
2 -> master: 216
1 -> master: 125
master -> 1:  7
master -> 2:  8
2 -> master: 512
1 -> master: 343
master -> 1:  9
master -> 2:  10
2 -> master: 1000
1 -> master: 729
master -> 1:  11
master -> 2:  12
2 -> master: 1728
1 -> master: 1331
master -> 1:  13
master -> 2:  14
2 -> master: 2744
1 -> master: 2197
master -> 1:  15
master -> 2:  16
2 -> master: 4096
1 -> master: 3375
master -> 1:  17
master -> 2:  18
2 -> master: 5832
1 -> master: 4913
master -> 1:  19
master -> 2:  20
2 -> master: 8000
1 -> master: 6859
master -> 1:  21
master -> 2:  22
2 -> master: 10648
1 -> master: 9261
master -> 1:  23
master -> 2:  24
2 -> master: 13824
1 -> master: 12167
master -> 1:  25
1 -> master: 15625
[ 1, 8, 27, 64, 125, 216, 343, 512, 729, 1000, 1331, 1728, 2197, 2744, 3375, 
  4096, 4913, 5832, 6859, 8000, 9261, 10648, 12167, 13824, 15625 ]
gap> ParInstallTOPCGlobalFunction( "MyParListWithAglom",
> function( list, fnc, aglomCount )
>   local result, iter;
>   result := [];
>   iter := Iterator(list);
>   MasterSlave( function() if IsDoneIterator(iter) then return NOTASK;
>                           else return NextIterator(iter); fi; end,
>                fnc,
>                function(input,output)
>                  local i;
>                  for i in [1..Length(input)] do
>                    result[input[i]] := output[i];
>                  od;
>                  return NO_ACTION;
>                end,
>                Error,  # Never called, can specify anything
>                aglomCount
>              );
>   return result;
> end );
gap> MyParListWithAglom( [1..25], x->x^3, 4 );
master -> 1: (AGGLOM_TASK): [ 1, 2, 3, 4 ]
master -> 2: (AGGLOM_TASK): [ 5, 6, 7, 8 ]
1 -> master: [ 1, 8, 27, 64 ]
2 -> master: [ 125, 216, 343, 512 ]
master -> 1: (AGGLOM_TASK): [ 9, 10, 11, 12 ]
master -> 2: (AGGLOM_TASK): [ 13, 14, 15, 16 ]
1 -> master: [ 729, 1000, 1331, 1728 ]
2 -> master: [ 2197, 2744, 3375, 4096 ]
master -> 1: (AGGLOM_TASK): [ 17, 18, 19, 20 ]
master -> 2: (AGGLOM_TASK): [ 21, 22, 23, 24 ]
1 -> master: [ 4913, 5832, 6859, 8000 ]
2 -> master: [ 9261, 10648, 12167, 13824 ]
master -> 1: (AGGLOM_TASK): [ 25 ]
1 -> master: [ 15625 ]
[ 1, 8, 27, 64, 125, 216, 343, 512, 729, 1000, 1331, 1728, 2197, 2744, 3375, 
  4096, 4913, 5832, 6859, 8000, 9261, 10648, 12167, 13824, 15625 ]
\endexample

If you wish  an  accelerated  introduction  to  the  models  of  parallel
programming provided here, you  might  wish  to  read  the  beginning  of
Chapter~"Slave Listener" through section~"Slave Listener  Commands",  and
then proceed immediately to Chapter~"Basic Concepts for the  TOP-C  model
(MasterSlave)".

%%%%%%%%%%%%%%%%%%%%%%%%%%%%%%%%%%%%%%%%%%%%%%%%%%%%%%%%%%%%%%%%%%%%%%%%%
\Section{Author}

The {\ParGAP} package was designed and written by Gene Cooperman, College
of Computer Science, Northeastern University, Boston, MA, U.S.A.

If you use {\ParGAP} to solve a problem then please send a short email to
\Mailto{gene@ccs.neu.edu} about it, and cite  the  {\ParGAP}  package  as
follows:

\begintt
\bibitem[Coo99]{Coo99}
      Cooperman, Gene,
      {\sl Parallel GAP/MPI (ParGAP/MPI)}, Version 1,
      College of Computer Science, Northeastern University, 1999,
      \verb+http://www.ccs.neu.edu/home/gene/pargap.html+.
\endtt

%%%%%%%%%%%%%%%%%%%%%%%%%%%%%%%%%%%%%%%%%%%%%%%%%%%%%%%%%%%%%%%%%%%%%%%%%
\Section{Invoking ParGAP with Remote Slaves (when using a system MPI library)}

{\ParGAP} can be built to use either a system MPI library, or the included
MPINU library. The command to run {\ParGAP} is different in the two cases.
If {\ParGAP} has been built using MPINU then you should skip this section
and proceed to section~"Invoking ParGAP with Remote Slaves (using MPINU)".
Otherwise, please read on.

After {\ParGAP} has been installed, a script `bin/pargap.sh' will have been
created  which   (after   any   changes   you   needed   to   make;   see
Section~"Installing ParGAP") you should use to invoke {\ParGAP}. Installers 
are encouraged to treat `pargap.sh' in analogy to `gap.sh'. For example, if 
your site  has  copied  `gap.sh'  to `/usr/local/bin/gap', then you should
also look for the `pargap.sh' script as `/usr/local/bin/pargap'. It simplifies
the remoste slave configuration if {\ParGAP} can be found on the standard
path on each machine, and we'll assume that in this section {\ParGAP} can
be invoked simply as `pargap'.

When built with a system MPI installation, {\ParGAP} must be invoked using
the system's MPI launcher. This may go under several names, but the command
name `mpiexec' is suggested in the MPI-2 specification, and is supported by 
both Open MPI and MPICH, two common implementations of that specification.

The basic usage is

\){\kernttindent}mpiexec -n <num> pargap

to launch `<num>' copies of {\ParGAP} (i.e. one master and $(num-1)$ slaves). 
With no other parameters, these will all be launched on the host machine.

A configuration file can be used to specify hosts for remote slaves. The
syntax of this file different for Open MPI and MPICH, but in both cases 
the configuration file is a text file listing the host names
and the number of processes to run on each host, one per line. The default 
number of processes per node is one by default.

When using Open MPI, an example <hostfile> is
\begintt
# Example Open MPI hostfile.  Comments begin with #
#
# The following node is a single processor machine:
foo.example.com
# The following two nodes are dual-processor machines:
bar.example.com slots=2
yow.example.com slots=2
\endtt
This hostfile is passed to `mpiexec' using

\){\kernttindent}mpiexec -n <num> -hostfile <hostfile> pargap

Processes are allocated round-robin style. For example, if we choose <num> to
be seven then the first process (the master) will run on `foo'. The
slaves will run two on `bar', two on `yow' and a further one each on `foo' and
`bar'. 

When using MPICH, the equivalent <machinefile> is
\begintt
# Example MPICH machinefile.  Comments begin with #
#
# The following node is a single processor machine:
foo.example.com
# The following two nodes are dual-processor machines:
bar.example.com:2
yow.example.com:2
\endtt
and the command to start {\ParGAP} using these hosts will be 

\){\kernttindent}mpiexec -n <num> -machinefile <machinefile> pargap

For further information, such as specifying hosts on the command line, or finer
control of how processes are distributed between hosts, or if you have a 
different MPI implementation, then please see your MPI documentation.

Unless you have any problems with the installation or running ParGAP,
you can skip the rest of this chapter and move on to Chapter~"Slave Listener".

%%%%%%%%%%%%%%%%%%%%%%%%%%%%%%%%%%%%%%%%%%%%%%%%%%%%%%%%%%%%%%%%%%%%%%%%%
\Section{Invoking ParGAP with Remote Slaves (when using MPINU)}

If {\ParGAP} has been built to use the supplied MPINU library then
{\ParGAP} includes the facility (on Linux) to start up and manage 
remove slaves without needing an external MPI launcher. If {\ParGAP} is 
built using a system MPI library then please read to 
section~"Invoking ParGAP with Remote Slaves (when using a system MPI library)"
instead. 

We'll assume that when {\ParGAP} was built the scipt `bin/pargap.sh' was
copied to `/usr/local/bin/pargap' (see Section~"Installing ParGAP").
{\ParGAP} can then be run by calling `pargap'. In addition, there  must  be  
a file, `procgroup', in the current directory, or alternatively, if you wish
to use a single procgroup file for all jobs, and that procgroup file is in  
`/home/joe', then you can alias `pargap' to `pargap -p4pg /home/joe/procgroup'.

The  procgroup  file  has  a  simple  syntax,  taken from the MPICH 
(not MPICH2) implementation of MPI. A `\#' in column~1  introduces
a comment line. The first non-comment line should be `local 0', verbatim.
This line declares the master process as the local process.  Other  lines
are of the form:

\){\kernttindent}<host-machine> 1 <pargap-script>

e.g.

\begintt
regulus.ccs.neu.edu 1 /usr/local/bin/pargap
\endtt

The first field is the hostname for a remote process.  The  second  field
specifies one thread per process. ({\ParGAP} recognizes only the  value~1
for the second field.) The  third  field  is  an  absolute  pathname  for
{\ParGAP}, as it would be called on the remote process. Note that you can
repeat the same line twice if you want two remote {\ParGAP} processes  on
the same processor. The default `procgroup' provided in the  distribution
will have lines of form:

\){\kernttindent}localhost 1 <path-of-provided-pargap.sh>

If you change <path-of-provided-pargap.sh> to just, say,  `pargap',  this
will work only if `pargap' is in your path on the  remote  machine  shell
(`localhost' in this case), using your default shell. On  most  machines,
`localhost' is an alias for the local processor. This is a  good  default
for debugging, so that you don't disturb users on other machines.

MPI will use a line

\){\kernttindent}<host-machine> 1 <pargap-script>

to create a UNIX subprocess executing:

\){\kernttindent}ssh <host-machine> <pargap-script>

Suppose <host-machine> is `regulus.ccs.neu.edu'  and  <pargap-script>  is
`/usr/local/bin/pargap' as in the above example,  and  we  were  to  have
trouble invoking {\ParGAP}, then it would be a good idea to try  invoking
`ssh regulus.ccs.neu.edu' from a UNIX prompt and  if  that  succeeds,  to
then try executing the full `ssh' command.

A typical problem is that the remote processor  requires  a  password  to
login.   MPI   requires   a   login   without    passwords.    This   can
be set up for `ssh'.  See `man ssh'. Sometimes, PAM is also used for user
authentication (see `/etc/pam.conf').  Consult  your  system  staff   for
further analysis. If your site uses an alternative to `ssh', there  is  a
solution here: add the lines

\begintt
#############################################################################
##
##  SSH . . . .. . . . . . . . . . . . . . . . .  remote shell used by ParGAP
##
##
SSH=myssh
export SSH
\endtt

before the `GAP' block with the `exec' line. (Of course, the  `\#'  lines
are not needed; they are comments.)

Note that the remote {\ParGAP} process will not read from standard input,
although signals such as SIGINT (`\^{}C') may be received by  the  remote
process. However, the remote {\ParGAP} process  will  write  to  standard
output, which is relayed to the local process. So,

\beginexample
gap> SendMsg("Exec(\"hostname\")", 2);
\endexample

will execute and print from the remote process.

%%%%%%%%%%%%%%%%%%%%%%%%%%%%%%%%%%%%%%%%%%%%%%%%%%%%%%%%%%%%%%%%%%%%%%%%%
\Section{Problems Installing or Invoking ParGAP}

If you still have problems, here is a list of things to check. This section
considers general problems when installing or running {\ParGAP}. The two 
sections after this one consider problems specific to using MPINU or a system
MPI library respectively.

\beginlist
\item{0.}
    If you are using {\ParGAP} on a Mac with MPINU2 or a system MPI 
    implementation then {\ParGAP} may consistently crash on startup. If this
    is the case then try using MPINU instead by reconfiguring {\ParGAP} with

\){\kernttindent}./configure --with-mpi=MPINU

\item{}
    This is a known issue which will be fixed in a forthcoming version.

\item{1.}
    Do you have enough swap space to support multiple {\GAP} processes? A
    simple way to check this is with the UNIX command, `top'.  The  Linux
    version of `top' sorts by memory usage if you type `M'.

\item{2.}
   `make' tries to automatically create:

\begintt
pkg/pargap/bin/pargap.sh
\endtt

\item{}
    and copy the parameters from `<GAP_ROOT>/bin/gap.sh'. <GAP_ROOT>  was
    specified when  you  executed  `./configure  <GAP_ROOT>'  to  install
    ParGAP. This can be error-prone if your site has an unusual setup. If
    you execute `<GAP_ROOT>/bin/gap.sh', does gap come up? If so, compare
    it   with   `pargap.sh'   and   check   for   correct   settings   in
    `.../pkg/pargap/bin/pargap.sh'?

\item{3.}
    Were the remote slave processes able to start up? If so,  could  they
    connect back to  the  master?  To  test  connectivity  problems,  try
    manually starting a remote slave by executing a line in  the  script.
    Try a simple `ssh <remote-hostname>' to see  if  the  issue  is  with
    security. If your site uses `ssh' instead of `ssh', then there  is  a
    security issue. Read Section~"Problems with Passwords (Getting Around
    Security)", and possibly `man sshd'.

\item{4.}
    If  the  previous  step  failed  due  to  security  issues,  such  as
    requesting a password, you have several options. `man ssh' tells  you
    the security model at your site.  Then read Section~"Problems with
    Passwords (Getting Around Security)".

\item{5.}
    Is `pargap' listed in `.../pkg/ALLPKG'?
    [It's needed to autostart slaves.]

\item{6.}
    Inside {\ParGAP}, has MPI been successfully initialized?
    Try:  
    
\beginexample
gap> MPI_Initialized();
\endexample

\item{7.}
    A remote (slave) {\ParGAP} process starts in your home directory  and
    tries to `cd'  to  a  directory  of  the  same  name  as  your  local
    directory. Check your assumptions about the remote machine. Try:

\beginexample
gap> SendRecvMsg("Exec(pwd)"); SendRecvMsg("UNIX_Hostname()");
gap> SendRecvMsg("UNIX_Getpid()");
\endexample

\item{8.}
    Every {\ParGAP} slave process displays its {\GAP} banner and startup 
    messages on the terminal of the master process. If you have many slaves
    and do not wish to see these messages, then pass the `-b' and/or `-q' 
    switches to {\ParGAP} when it starts, to disable the banner or all messages
    respectively. See Section~"Ref:Command Line Options" of the GAP Reference
    Manual for further details.

\item{9.}
    Read the documentation for further possible problems.

\endlist

%%%%%%%%%%%%%%%%%%%%%%%%%%%%%%%%%%%%%%%%%%%%%%%%%%%%%%%%%%%%%%%%%%%%%%%%%
\Section{Problems Running ParGAP with MPINU}

If you have problems running {\ParGAP}, and {\ParGAP} is built to use the 
supplied MPINU library, then this section lists some things to check, in 
addition to the general issues listed in the previous section. If you are
using a system MPI implementation instead of MPINU, this section can be 
ignored, but you should read the next section instead.


\beginlist
\item{1.}
    Did {\ParGAP} find your `procgroup' file?
    [It looks in the current directory for `procgroup', or for:

\){\kernttindent}... -p4pg <PATH>/procgroup

\item{}
    on the command line.]

\item{2.}
    If you are using MPINU, is the `procgroup' file in your current directory 
    set correctly? Test it. If you are calling it on a remote host, manually 
    type:

\){\kernttindent}ssh <HOSTNAME> <ParGAP>

\item{}
    where <HOSTNAME> and <ParGAP> appear exactly as in `procgroup', e.g.
    
\){\kernttindent}ssh denali.ccs.neu.edu /usr/local/gap4r3/bin/pargap.sh

\item{}
    In some cases, `exec' is used to save process overhead. Also try:

\){\kernttindent}ssh <HOSTNAME> exec <ParGAP>

\item{}
    If you plan to call it on localhost, try just:   <ParGAP>

\item{}
    Note that if not all the slave processes succeed in connecting
    to the master, then {\ParGAP} writes out a file:

\begintt
/tmp/pargapmpi-ssh.xx
\endtt
       
\item{}
    where `xx' is replaced by the  the  process  id  of  the  {\ParGAP}
    process.

\item{3.}
    If the connection dies at random, after some period of time:
    You can experiment with `SO_KEEPALIVE' and variants.  
    (See `man setsockopt'.)
    This periodically sends *null messages* so the  remote  machine  does
    not think that the originating  machine  is  dead.  However,  if  the
    remote machine fails to reply, the  local  process  sends  a  SIGPIPE
    signal to notify current processes of a broken  socket,  even  though
    there might have been only a temporary lapse in connectivity.
    `ssh' specifies `KeepAlive yes' by default, but setting `KeepAlive no'
    might get you through some transient lapses in  connectivity  due  to
    high congestion. 
    You may also want to experiment with: `setenv SSH "ssh -n"'

\item{4.}
    If a host is on multiple networks, it will have multiple IP addresses and
    usually multiple hostnames. In  this  case,  the  master  process  cannot
    always guess correctly which IP address (which internet  address)  should
    be passed to the slave process, so that the slave process can  call  back
    to the master. In such cases,  you  may  need  to  tell  {\ParGAP}  which
    hostname or IP address to use for the callback. This is done  by  setting
    the UNIX environment variable, `CALLBACK_HOST', as in the example below.

\begintt
# [ in sh/bash/... ]
CALLBACK_HOST=denali.ccs.neu.edu; export CALLBACK_HOST
# [ in csh/tcsh/... ]
setenv CALLBACK_HOST=denali.ccs.neu.edu
\endtt

\item{}
    The appropriate  line  for  your  shell  can  be  placed  in  your  shell
    initialization file. Alternatively, you can set this up for all users  by
    placing the Bourne shell version (for `sh') somewhere between  the  first
    and last line of `.../pkg/pargap/bin/pargap.sh'.

\item{5.}
    {\ParGAP} is supplied with two different versions of MPINU: the original
    MPINU and a later version, MPINU2, and it will also work with other MPI
    libraries if they are present on your system. By default, if you do
    not have a system MPI implementation then MPINU2 is used. If you have 
    problems which appear to be MPI-related, try rebuilding {\ParGAP} with a 
    different MPI library. For example, to use MPINU instead of MPINU2 then
    run configure using
\begintt
./configure --with-mpi=MPINU
\endtt

\endlist


%%%%%%%%%%%%%%%%%%%%%%%%%%%%%%%%%%%%%%%%%%%%%%%%%%%%%%%%%%%%%%%%%%%%%%%%%
\Section{Problems Running ParGAP with a System MPI Implementation}

Here are a list of known issues when using a system MPI library with {\ParGAP}, 
and some solutions or workarounds. Not all of these issues will manifest 
themselves on all architectures and all MPI implementations. If you are having 
problems building or running {\ParGAP}, you should check this section as well as 
Section~"Problems Installing or Invoking ParGAP"

\beginlist
\item{1.}
  Line editing at the GAP command prompt is unlikely to work when {\ParGAP} is
  invoked with an MPI launcher, since they tend to do their own processing 
  of the terminal I/O (stdin/stdout/stderr) which does not work well 
  either the readline library used in newer versions of {\GAP} or the in-built
  terminal editing in earlier versions of {\GAP}. It may be useful to run
  {\ParGAP} through the `rlwrap' utility, if available. For example, if 
  {\ParGAP} is run using `mpiexec', then try

\begintt
rlwrap mpiexec -n 3 pargap
\endtt

\item{}
  This should restore some of the line editing, although tab completion is 
  limited to commands that `rlwrap' has already seen you use. For more 
  information, try `man rlwrap'.

\item{2.}
  The command `FlushAllMsgs()' (see~"FlushAllMsgs") is not available 
  when using a system MPI implementation, since it tests show that 
  `ProbeMsgNonBlocking()', which it uses (see~"ProbeMsgNonBlocking") cannot be 
  relied upon to always return `true' the first time that it is called after a
  message has been sent. If your system MPI implementation does exhibit this 
  desired behaviour for `ProbeMsgNonBlocking()' then you can install your own
  local copy of `FlushAllMsgs()' by copying the code for this function from 
  `lib/slavelist.g', removing the `if' statement and renaming the function.

\item{3.}
  The command `ParReset()' (see~"ParReset") is not available when using a
  system MPI implementation. When using a MPINU library, the slaves are launched
  by {\ParGAP} itself and so can be contacted and restarted, but with a system
  MPI library the slaves are launched by `mpiexec' (or whichever MPI launcher
  you use) and so cannot be reset from within {\ParGAP}. There is no known
  workaround for this.

\item{4.}
  {\GAP} and, in particular, the \package{IO} Package install handlers for the 
  SIGCHLD signal. Many implementations of MPI also install their own SIGCHLD 
  handler, which may then conflict with {\ParGAP}. Testing has revealed no 
  issues, but we cannot guarantee that there will be no interaction between the 
  two. In particular, this may result in temporary files not being cleaned up 
  properly.

\item{5.}
  The {\GAP} memory manager, GASMAN, can run into problems extending the 
  {\GAP} workspace if external libraries use `malloc' to allocate their own
  memory. MPINU avoids the use of `malloc' as much as possible, but
  system MPI implementations may not be as careful. This can be resolved by
  starting {\ParGAP} with the `-s' command-line switch, which asks
  {\ParGAP} to pre-allocate memory before it starts. You can safely 
  pre-allocate more memory than you will actually need since physical memory
  will only be mapped when it is actually used, so for example you could 
  allocate 3Gb:
\begintt
mpiexec -n 3 pargap -s 3g
\endtt
  The `-a' and `-m' switches can also be used to control memory usage. See 
  Section~"Ref:Command Line Options" of the GAP Reference Manuel for 
  further information.
\endlist

News of any other issues or solutions would be gratefully accepted.



%%%%%%%%%%%%%%%%%%%%%%%%%%%%%%%%%%%%%%%%%%%%%%%%%%%%%%%%%%%%%%%%%%%%%%%%%
\Section{Problems with Passwords (Getting Around Security)}

There is a simple test to see if you need to read this  section.  Pick  a
remote machine, <HOSTNAME>, that you wish to execute on, and  type:  `ssh
<HOSTNAME>'. If this did not work, also try `ssh <HOSTNAME>'. If you were
asked for your password, then you and your system administrator may  need
to talk about security policy. If you were successful with an alternative
to `ssh' then  set  the environment  variable, `SSH', to  the alternative
value, as described in item~3 below.

\beginlist
\item{(1)}
    Add a `.shosts' file to your home directory (for `ssh').

\item{(2)}
    Hack around the problem: By default, the startup script uses `ssh' to
    start remote processes. However, if the  environment  variable  `SSH'
    was set, the script  uses  the  value  of  the  environment  variable
    instead of `ssh'. This may be useful, if you have  your  own  script,
    `myssh', that automatically gets around  the  security  issues.  Then
    just type:

\begintt
SSH=myrsh; export SSH  # [ in sh/bash/... ]
setenv SSH myrsh       # [ in csh/tcsh/... ]
\endtt

\item{}
    The appropriate line for your shell  can  be  placed  in  your  shell
    initialization file. Alternatively, you can set this up for all users
    by placing the Bourne shell version (for `sh') somewhere between  the
    first and last line of `.../pkg/pargap/bin/pargap.sh'.  (The  example
    for `ssh' was given earlier.)

\item{(3)}
    `ssh': `man ssh' mentions some possibilities for giving the  password
    the first time, and then having ssh remember that  future  logins  to
    that machine are authorized for the duration of  the  session.  Don't
    overlook the use of `\$HOME/.ssh/config' to set  special  parameters,
    such as specifying a different login name on the remote machine. Some
    parameters of interest  might  be  `KeepAlive',  `RSAAuthentication',
    `UseRsh'. You may also find useful information in `man sshd'.

\item{(4)}
     After starting {\ParGAP}, manually call

\begintt
/tmp/pargapmpi-ssh.$$
\endtt

\item{}
    and repeatedly type in the password for each slave  process.  If  you
    find yourself doing this, you may  want  to  talk  with  your  system
    administrator, since it actually hurts system security  to  have  you
    repeatedly typing passwords with a  concommitant  risk  that  someone
    else will find out your password.

\endlist



%%%%%%%%%%%%%%%%%%%%%%%%%%%%%%%%%%%%%%%%%%%%%%%%%%%%%%%%%%%%%%%%%%%%%%%%%
\Section{Modifying the GAP kernel}

Note that this package modifies the {\GAP} `src'  and  `bin'  files,  and
creates a new {\GAP} kernel. This new {\GAP}  kernel  can  be  shared  by
traditional users of the old, sequential  {\GAP}  kernel,  and  by  those
doing parallel processing.

The {\GAP} kernel will have identical behavior to the old  {\GAP}  kernel
when invoked through  the  `gap.sh'  script  or  the  `bin/@GAParch@/gap'
binary. The new {\ParGAP} variables will appear to the end user *ONLY* if
the {\GAP} binary was invoked as `pargapmpi':  a  symbolic  link  to  the
actual {\GAP} binary. The script, `pargap.sh', does this.

So, in a multi-user environment, traditional users can  continue  to  use
`gap.sh'  without  noticing  any  difference.  Only  an   invocation   of
`pargap.sh' will add the new features.

In a future version of {\GAP}, it is hoped that the  {\GAP}  kernel  will
have enough ``hooks'', so that no modification of the  {\GAP}  kernel  is
required. At that time, it will also be possible to speed up the  startup
time for {\ParGAP}. Much of the startup time is  caused  by  waiting  for
{\GAP} to read its library files. It will be possible to use  the  {\GAP}
function, `SaveWorkspace()' to save a version  with  the  {\GAP}  library
pre-loaded. That saved version can then be used to  start  up  {\ParGAP}.
This is not currently possible, because {\ParGAP} needs  to  get  at  the
command line of {\GAP} before the {\GAP} kernel sees it.

Comments and contributions to a {\ParGAP} user library, or any other type
of assistance, are gratefully accepted.


Gene Cooperman
\Mailto{gene@ccs.neu.edu}

%%%%%%%%%%%%%%%%%%%%%%%%%%%%%%%%%%%%%%%%%%%%%%%%%%%%%%%%%%%%%%%%%%%%%%%%%
%%
%E
